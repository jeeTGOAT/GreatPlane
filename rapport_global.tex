\documentclass[12pt,a4paper]{article}
\usepackage[utf8]{inputenc}
\usepackage[T1]{fontenc}
\usepackage[french]{babel}
\usepackage{graphicx}
\usepackage{amsmath}
\usepackage{listings}
\usepackage{xcolor}
\usepackage{hyperref}
\usepackage{enumitem}
\usepackage{geometry}
\usepackage{fancyhdr}

\geometry{a4paper, margin=2.5cm}

% Fix the header height
\setlength{\headheight}{16.35004pt}
\addtolength{\topmargin}{-0.996pt}

\pagestyle{fancy}
\fancyhf{}
\fancyhead[L]{\includegraphics[height=12pt]{n7.png}}
\fancyhead[C]{Équipe KL-4}
\fancyhead[R]{\thepage}
\fancyfoot[C]{Simulateur d'Avion - 2025}

% Configuration des listings pour Java
\definecolor{javared}{rgb}{0.6,0,0}
\definecolor{javagreen}{rgb}{0.25,0.5,0.35}
\definecolor{javapurple}{rgb}{0.5,0,0.35}
\definecolor{javacomment}{rgb}{0.5,0.5,0.5}

\lstset{
    language=Java,
    basicstyle=\ttfamily\small,
    keywordstyle=\color{javapurple}\bfseries,
    stringstyle=\color{javared},
    commentstyle=\color{javacomment},
    numbers=left,
    numberstyle=\tiny\color{javacomment},
    stepnumber=1,
    numbersep=8pt,
    showspaces=false,
    showstringspaces=false,
    breaklines=true,
    frameround=ftff,
    frame=lines,
    backgroundcolor=\color{white},
    captionpos=b
}

\title{
    \includegraphics[width=0.4\textwidth]{n7.png}\\[1cm]
    \textbf{Rapport Global - Projet Simulateur d'Avion}
}
\author{Équipe de Développement}
\date{\today}

\begin{document}

\maketitle
\thispagestyle{empty}

\tableofcontents
\newpage

\section{Introduction}
Ce rapport présente l'ensemble du projet de développement du Simulateur d'Avion, incluant les modifications récentes apportées à l'interface utilisateur et les nouvelles fonctionnalités implémentées.

\section{Contexte du Projet}
Le Simulateur d'Avion est une application Java permettant de simuler le trafic aérien en temps réel. Le projet a évolué pour inclure de nouvelles fonctionnalités et améliorer l'expérience utilisateur.

\section{Évolutions Récentes}
\subsection{Amélioration de l'Interface Utilisateur}
\begin{itemize}
    \item Réorganisation des boutons du menu principal
    \item Ajout du bouton "HISTORIQUE DES VOLS"
    \item Repositionnement du bouton "QUITTER"
    \item Amélioration de l'ergonomie générale
\end{itemize}

\subsection{Nouvelles Fonctionnalités}
\begin{itemize}
    \item Système d'historique des vols
    \item Visualisation du trafic aérien
    \item Interface de simulation améliorée
\end{itemize}

\section{Architecture Technique}
\subsection{Structure du Code}
\begin{itemize}
    \item MenuPage.java : Gestion de l'interface principale
    \item SimulationPanel.java : Logique de simulation
    \item FlightHistoryPanel.java : Gestion de l'historique
    \item Main.java : Point d'entrée de l'application
\end{itemize}

\section{Améliorations Apportées}
\subsection{Interface Utilisateur}
\begin{itemize}
    \item Disposition plus intuitive des boutons
    \item Meilleure séparation des fonctionnalités
    \item Interface plus professionnelle
\end{itemize}

\subsection{Fonctionnalités}
\begin{itemize}
    \item Ajout du système d'historique
    \item Amélioration de la simulation
    \item Meilleure gestion des interactions
\end{itemize}

\section{Documentation}
\subsection{Manuel Utilisateur}
Un manuel utilisateur complet a été créé pour guider les utilisateurs dans l'utilisation de l'application.

\subsection{Rapports Techniques}
\begin{itemize}
    \item Rapport individuel détaillant les modifications
    \item Documentation du code
    \item Guide d'installation
\end{itemize}

\section{Conclusion}
Le projet a connu des améliorations significatives, notamment dans son interface utilisateur et ses fonctionnalités. Les modifications apportées ont permis de créer une application plus intuitive et plus complète.

\section{Perspectives}
\begin{itemize}
    \item Amélioration continue de l'interface
    \item Ajout de nouvelles fonctionnalités
    \item Optimisation des performances
\end{itemize}

\end{document} 